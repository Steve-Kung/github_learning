
% Default to the notebook output style

    


% Inherit from the specified cell style.




    
\documentclass[11pt]{article}

    
    
    \usepackage[T1]{fontenc}
    % Nicer default font (+ math font) than Computer Modern for most use cases
    \usepackage{mathpazo}

    % Basic figure setup, for now with no caption control since it's done
    % automatically by Pandoc (which extracts ![](path) syntax from Markdown).
    \usepackage{graphicx}
    % We will generate all images so they have a width \maxwidth. This means
    % that they will get their normal width if they fit onto the page, but
    % are scaled down if they would overflow the margins.
    \makeatletter
    \def\maxwidth{\ifdim\Gin@nat@width>\linewidth\linewidth
    \else\Gin@nat@width\fi}
    \makeatother
    \let\Oldincludegraphics\includegraphics
    % Set max figure width to be 80% of text width, for now hardcoded.
    \renewcommand{\includegraphics}[1]{\Oldincludegraphics[width=.8\maxwidth]{#1}}
    % Ensure that by default, figures have no caption (until we provide a
    % proper Figure object with a Caption API and a way to capture that
    % in the conversion process - todo).
    \usepackage{caption}
    \DeclareCaptionLabelFormat{nolabel}{}
    \captionsetup{labelformat=nolabel}

    \usepackage{adjustbox} % Used to constrain images to a maximum size 
    \usepackage{xcolor} % Allow colors to be defined
    \usepackage{enumerate} % Needed for markdown enumerations to work
    \usepackage{geometry} % Used to adjust the document margins
    \usepackage{amsmath} % Equations
    \usepackage{amssymb} % Equations
    \usepackage{textcomp} % defines textquotesingle
    % Hack from http://tex.stackexchange.com/a/47451/13684:
    \AtBeginDocument{%
        \def\PYZsq{\textquotesingle}% Upright quotes in Pygmentized code
    }
    \usepackage{upquote} % Upright quotes for verbatim code
    \usepackage{eurosym} % defines \euro
    \usepackage[mathletters]{ucs} % Extended unicode (utf-8) support
    \usepackage[utf8x]{inputenc} % Allow utf-8 characters in the tex document
    \usepackage{fancyvrb} % verbatim replacement that allows latex
    \usepackage{grffile} % extends the file name processing of package graphics 
                         % to support a larger range 
    % The hyperref package gives us a pdf with properly built
    % internal navigation ('pdf bookmarks' for the table of contents,
    % internal cross-reference links, web links for URLs, etc.)
    \usepackage{hyperref}
    \usepackage{longtable} % longtable support required by pandoc >1.10
    \usepackage{booktabs}  % table support for pandoc > 1.12.2
    \usepackage[inline]{enumitem} % IRkernel/repr support (it uses the enumerate* environment)
    \usepackage[normalem]{ulem} % ulem is needed to support strikethroughs (\sout)
                                % normalem makes italics be italics, not underlines
    

    
    
    % Colors for the hyperref package
    \definecolor{urlcolor}{rgb}{0,.145,.698}
    \definecolor{linkcolor}{rgb}{.71,0.21,0.01}
    \definecolor{citecolor}{rgb}{.12,.54,.11}

    % ANSI colors
    \definecolor{ansi-black}{HTML}{3E424D}
    \definecolor{ansi-black-intense}{HTML}{282C36}
    \definecolor{ansi-red}{HTML}{E75C58}
    \definecolor{ansi-red-intense}{HTML}{B22B31}
    \definecolor{ansi-green}{HTML}{00A250}
    \definecolor{ansi-green-intense}{HTML}{007427}
    \definecolor{ansi-yellow}{HTML}{DDB62B}
    \definecolor{ansi-yellow-intense}{HTML}{B27D12}
    \definecolor{ansi-blue}{HTML}{208FFB}
    \definecolor{ansi-blue-intense}{HTML}{0065CA}
    \definecolor{ansi-magenta}{HTML}{D160C4}
    \definecolor{ansi-magenta-intense}{HTML}{A03196}
    \definecolor{ansi-cyan}{HTML}{60C6C8}
    \definecolor{ansi-cyan-intense}{HTML}{258F8F}
    \definecolor{ansi-white}{HTML}{C5C1B4}
    \definecolor{ansi-white-intense}{HTML}{A1A6B2}

    % commands and environments needed by pandoc snippets
    % extracted from the output of `pandoc -s`
    \providecommand{\tightlist}{%
      \setlength{\itemsep}{0pt}\setlength{\parskip}{0pt}}
    \DefineVerbatimEnvironment{Highlighting}{Verbatim}{commandchars=\\\{\}}
    % Add ',fontsize=\small' for more characters per line
    \newenvironment{Shaded}{}{}
    \newcommand{\KeywordTok}[1]{\textcolor[rgb]{0.00,0.44,0.13}{\textbf{{#1}}}}
    \newcommand{\DataTypeTok}[1]{\textcolor[rgb]{0.56,0.13,0.00}{{#1}}}
    \newcommand{\DecValTok}[1]{\textcolor[rgb]{0.25,0.63,0.44}{{#1}}}
    \newcommand{\BaseNTok}[1]{\textcolor[rgb]{0.25,0.63,0.44}{{#1}}}
    \newcommand{\FloatTok}[1]{\textcolor[rgb]{0.25,0.63,0.44}{{#1}}}
    \newcommand{\CharTok}[1]{\textcolor[rgb]{0.25,0.44,0.63}{{#1}}}
    \newcommand{\StringTok}[1]{\textcolor[rgb]{0.25,0.44,0.63}{{#1}}}
    \newcommand{\CommentTok}[1]{\textcolor[rgb]{0.38,0.63,0.69}{\textit{{#1}}}}
    \newcommand{\OtherTok}[1]{\textcolor[rgb]{0.00,0.44,0.13}{{#1}}}
    \newcommand{\AlertTok}[1]{\textcolor[rgb]{1.00,0.00,0.00}{\textbf{{#1}}}}
    \newcommand{\FunctionTok}[1]{\textcolor[rgb]{0.02,0.16,0.49}{{#1}}}
    \newcommand{\RegionMarkerTok}[1]{{#1}}
    \newcommand{\ErrorTok}[1]{\textcolor[rgb]{1.00,0.00,0.00}{\textbf{{#1}}}}
    \newcommand{\NormalTok}[1]{{#1}}
    
    % Additional commands for more recent versions of Pandoc
    \newcommand{\ConstantTok}[1]{\textcolor[rgb]{0.53,0.00,0.00}{{#1}}}
    \newcommand{\SpecialCharTok}[1]{\textcolor[rgb]{0.25,0.44,0.63}{{#1}}}
    \newcommand{\VerbatimStringTok}[1]{\textcolor[rgb]{0.25,0.44,0.63}{{#1}}}
    \newcommand{\SpecialStringTok}[1]{\textcolor[rgb]{0.73,0.40,0.53}{{#1}}}
    \newcommand{\ImportTok}[1]{{#1}}
    \newcommand{\DocumentationTok}[1]{\textcolor[rgb]{0.73,0.13,0.13}{\textit{{#1}}}}
    \newcommand{\AnnotationTok}[1]{\textcolor[rgb]{0.38,0.63,0.69}{\textbf{\textit{{#1}}}}}
    \newcommand{\CommentVarTok}[1]{\textcolor[rgb]{0.38,0.63,0.69}{\textbf{\textit{{#1}}}}}
    \newcommand{\VariableTok}[1]{\textcolor[rgb]{0.10,0.09,0.49}{{#1}}}
    \newcommand{\ControlFlowTok}[1]{\textcolor[rgb]{0.00,0.44,0.13}{\textbf{{#1}}}}
    \newcommand{\OperatorTok}[1]{\textcolor[rgb]{0.40,0.40,0.40}{{#1}}}
    \newcommand{\BuiltInTok}[1]{{#1}}
    \newcommand{\ExtensionTok}[1]{{#1}}
    \newcommand{\PreprocessorTok}[1]{\textcolor[rgb]{0.74,0.48,0.00}{{#1}}}
    \newcommand{\AttributeTok}[1]{\textcolor[rgb]{0.49,0.56,0.16}{{#1}}}
    \newcommand{\InformationTok}[1]{\textcolor[rgb]{0.38,0.63,0.69}{\textbf{\textit{{#1}}}}}
    \newcommand{\WarningTok}[1]{\textcolor[rgb]{0.38,0.63,0.69}{\textbf{\textit{{#1}}}}}
    
    
    % Define a nice break command that doesn't care if a line doesn't already
    % exist.
    \def\br{\hspace*{\fill} \\* }
    % Math Jax compatability definitions
    \def\gt{>}
    \def\lt{<}
    % Document parameters
    \title{github???}
    
    
    

    % Pygments definitions
    
\makeatletter
\def\PY@reset{\let\PY@it=\relax \let\PY@bf=\relax%
    \let\PY@ul=\relax \let\PY@tc=\relax%
    \let\PY@bc=\relax \let\PY@ff=\relax}
\def\PY@tok#1{\csname PY@tok@#1\endcsname}
\def\PY@toks#1+{\ifx\relax#1\empty\else%
    \PY@tok{#1}\expandafter\PY@toks\fi}
\def\PY@do#1{\PY@bc{\PY@tc{\PY@ul{%
    \PY@it{\PY@bf{\PY@ff{#1}}}}}}}
\def\PY#1#2{\PY@reset\PY@toks#1+\relax+\PY@do{#2}}

\expandafter\def\csname PY@tok@cm\endcsname{\let\PY@it=\textit\def\PY@tc##1{\textcolor[rgb]{0.25,0.50,0.50}{##1}}}
\expandafter\def\csname PY@tok@mo\endcsname{\def\PY@tc##1{\textcolor[rgb]{0.40,0.40,0.40}{##1}}}
\expandafter\def\csname PY@tok@s1\endcsname{\def\PY@tc##1{\textcolor[rgb]{0.73,0.13,0.13}{##1}}}
\expandafter\def\csname PY@tok@kp\endcsname{\def\PY@tc##1{\textcolor[rgb]{0.00,0.50,0.00}{##1}}}
\expandafter\def\csname PY@tok@si\endcsname{\let\PY@bf=\textbf\def\PY@tc##1{\textcolor[rgb]{0.73,0.40,0.53}{##1}}}
\expandafter\def\csname PY@tok@go\endcsname{\def\PY@tc##1{\textcolor[rgb]{0.53,0.53,0.53}{##1}}}
\expandafter\def\csname PY@tok@il\endcsname{\def\PY@tc##1{\textcolor[rgb]{0.40,0.40,0.40}{##1}}}
\expandafter\def\csname PY@tok@w\endcsname{\def\PY@tc##1{\textcolor[rgb]{0.73,0.73,0.73}{##1}}}
\expandafter\def\csname PY@tok@bp\endcsname{\def\PY@tc##1{\textcolor[rgb]{0.00,0.50,0.00}{##1}}}
\expandafter\def\csname PY@tok@sc\endcsname{\def\PY@tc##1{\textcolor[rgb]{0.73,0.13,0.13}{##1}}}
\expandafter\def\csname PY@tok@ch\endcsname{\let\PY@it=\textit\def\PY@tc##1{\textcolor[rgb]{0.25,0.50,0.50}{##1}}}
\expandafter\def\csname PY@tok@nc\endcsname{\let\PY@bf=\textbf\def\PY@tc##1{\textcolor[rgb]{0.00,0.00,1.00}{##1}}}
\expandafter\def\csname PY@tok@nb\endcsname{\def\PY@tc##1{\textcolor[rgb]{0.00,0.50,0.00}{##1}}}
\expandafter\def\csname PY@tok@sh\endcsname{\def\PY@tc##1{\textcolor[rgb]{0.73,0.13,0.13}{##1}}}
\expandafter\def\csname PY@tok@err\endcsname{\def\PY@bc##1{\setlength{\fboxsep}{0pt}\fcolorbox[rgb]{1.00,0.00,0.00}{1,1,1}{\strut ##1}}}
\expandafter\def\csname PY@tok@nn\endcsname{\let\PY@bf=\textbf\def\PY@tc##1{\textcolor[rgb]{0.00,0.00,1.00}{##1}}}
\expandafter\def\csname PY@tok@ss\endcsname{\def\PY@tc##1{\textcolor[rgb]{0.10,0.09,0.49}{##1}}}
\expandafter\def\csname PY@tok@mb\endcsname{\def\PY@tc##1{\textcolor[rgb]{0.40,0.40,0.40}{##1}}}
\expandafter\def\csname PY@tok@o\endcsname{\def\PY@tc##1{\textcolor[rgb]{0.40,0.40,0.40}{##1}}}
\expandafter\def\csname PY@tok@gi\endcsname{\def\PY@tc##1{\textcolor[rgb]{0.00,0.63,0.00}{##1}}}
\expandafter\def\csname PY@tok@nf\endcsname{\def\PY@tc##1{\textcolor[rgb]{0.00,0.00,1.00}{##1}}}
\expandafter\def\csname PY@tok@mi\endcsname{\def\PY@tc##1{\textcolor[rgb]{0.40,0.40,0.40}{##1}}}
\expandafter\def\csname PY@tok@cs\endcsname{\let\PY@it=\textit\def\PY@tc##1{\textcolor[rgb]{0.25,0.50,0.50}{##1}}}
\expandafter\def\csname PY@tok@mh\endcsname{\def\PY@tc##1{\textcolor[rgb]{0.40,0.40,0.40}{##1}}}
\expandafter\def\csname PY@tok@m\endcsname{\def\PY@tc##1{\textcolor[rgb]{0.40,0.40,0.40}{##1}}}
\expandafter\def\csname PY@tok@gs\endcsname{\let\PY@bf=\textbf}
\expandafter\def\csname PY@tok@mf\endcsname{\def\PY@tc##1{\textcolor[rgb]{0.40,0.40,0.40}{##1}}}
\expandafter\def\csname PY@tok@sx\endcsname{\def\PY@tc##1{\textcolor[rgb]{0.00,0.50,0.00}{##1}}}
\expandafter\def\csname PY@tok@s\endcsname{\def\PY@tc##1{\textcolor[rgb]{0.73,0.13,0.13}{##1}}}
\expandafter\def\csname PY@tok@vi\endcsname{\def\PY@tc##1{\textcolor[rgb]{0.10,0.09,0.49}{##1}}}
\expandafter\def\csname PY@tok@kc\endcsname{\let\PY@bf=\textbf\def\PY@tc##1{\textcolor[rgb]{0.00,0.50,0.00}{##1}}}
\expandafter\def\csname PY@tok@sb\endcsname{\def\PY@tc##1{\textcolor[rgb]{0.73,0.13,0.13}{##1}}}
\expandafter\def\csname PY@tok@c1\endcsname{\let\PY@it=\textit\def\PY@tc##1{\textcolor[rgb]{0.25,0.50,0.50}{##1}}}
\expandafter\def\csname PY@tok@cp\endcsname{\def\PY@tc##1{\textcolor[rgb]{0.74,0.48,0.00}{##1}}}
\expandafter\def\csname PY@tok@ni\endcsname{\let\PY@bf=\textbf\def\PY@tc##1{\textcolor[rgb]{0.60,0.60,0.60}{##1}}}
\expandafter\def\csname PY@tok@ge\endcsname{\let\PY@it=\textit}
\expandafter\def\csname PY@tok@ow\endcsname{\let\PY@bf=\textbf\def\PY@tc##1{\textcolor[rgb]{0.67,0.13,1.00}{##1}}}
\expandafter\def\csname PY@tok@s2\endcsname{\def\PY@tc##1{\textcolor[rgb]{0.73,0.13,0.13}{##1}}}
\expandafter\def\csname PY@tok@gh\endcsname{\let\PY@bf=\textbf\def\PY@tc##1{\textcolor[rgb]{0.00,0.00,0.50}{##1}}}
\expandafter\def\csname PY@tok@gd\endcsname{\def\PY@tc##1{\textcolor[rgb]{0.63,0.00,0.00}{##1}}}
\expandafter\def\csname PY@tok@gp\endcsname{\let\PY@bf=\textbf\def\PY@tc##1{\textcolor[rgb]{0.00,0.00,0.50}{##1}}}
\expandafter\def\csname PY@tok@kt\endcsname{\def\PY@tc##1{\textcolor[rgb]{0.69,0.00,0.25}{##1}}}
\expandafter\def\csname PY@tok@nd\endcsname{\def\PY@tc##1{\textcolor[rgb]{0.67,0.13,1.00}{##1}}}
\expandafter\def\csname PY@tok@kn\endcsname{\let\PY@bf=\textbf\def\PY@tc##1{\textcolor[rgb]{0.00,0.50,0.00}{##1}}}
\expandafter\def\csname PY@tok@kr\endcsname{\let\PY@bf=\textbf\def\PY@tc##1{\textcolor[rgb]{0.00,0.50,0.00}{##1}}}
\expandafter\def\csname PY@tok@no\endcsname{\def\PY@tc##1{\textcolor[rgb]{0.53,0.00,0.00}{##1}}}
\expandafter\def\csname PY@tok@na\endcsname{\def\PY@tc##1{\textcolor[rgb]{0.49,0.56,0.16}{##1}}}
\expandafter\def\csname PY@tok@cpf\endcsname{\let\PY@it=\textit\def\PY@tc##1{\textcolor[rgb]{0.25,0.50,0.50}{##1}}}
\expandafter\def\csname PY@tok@nl\endcsname{\def\PY@tc##1{\textcolor[rgb]{0.63,0.63,0.00}{##1}}}
\expandafter\def\csname PY@tok@vg\endcsname{\def\PY@tc##1{\textcolor[rgb]{0.10,0.09,0.49}{##1}}}
\expandafter\def\csname PY@tok@gr\endcsname{\def\PY@tc##1{\textcolor[rgb]{1.00,0.00,0.00}{##1}}}
\expandafter\def\csname PY@tok@gt\endcsname{\def\PY@tc##1{\textcolor[rgb]{0.00,0.27,0.87}{##1}}}
\expandafter\def\csname PY@tok@vc\endcsname{\def\PY@tc##1{\textcolor[rgb]{0.10,0.09,0.49}{##1}}}
\expandafter\def\csname PY@tok@gu\endcsname{\let\PY@bf=\textbf\def\PY@tc##1{\textcolor[rgb]{0.50,0.00,0.50}{##1}}}
\expandafter\def\csname PY@tok@nv\endcsname{\def\PY@tc##1{\textcolor[rgb]{0.10,0.09,0.49}{##1}}}
\expandafter\def\csname PY@tok@c\endcsname{\let\PY@it=\textit\def\PY@tc##1{\textcolor[rgb]{0.25,0.50,0.50}{##1}}}
\expandafter\def\csname PY@tok@nt\endcsname{\let\PY@bf=\textbf\def\PY@tc##1{\textcolor[rgb]{0.00,0.50,0.00}{##1}}}
\expandafter\def\csname PY@tok@sr\endcsname{\def\PY@tc##1{\textcolor[rgb]{0.73,0.40,0.53}{##1}}}
\expandafter\def\csname PY@tok@kd\endcsname{\let\PY@bf=\textbf\def\PY@tc##1{\textcolor[rgb]{0.00,0.50,0.00}{##1}}}
\expandafter\def\csname PY@tok@k\endcsname{\let\PY@bf=\textbf\def\PY@tc##1{\textcolor[rgb]{0.00,0.50,0.00}{##1}}}
\expandafter\def\csname PY@tok@se\endcsname{\let\PY@bf=\textbf\def\PY@tc##1{\textcolor[rgb]{0.73,0.40,0.13}{##1}}}
\expandafter\def\csname PY@tok@ne\endcsname{\let\PY@bf=\textbf\def\PY@tc##1{\textcolor[rgb]{0.82,0.25,0.23}{##1}}}
\expandafter\def\csname PY@tok@sd\endcsname{\let\PY@it=\textit\def\PY@tc##1{\textcolor[rgb]{0.73,0.13,0.13}{##1}}}

\def\PYZbs{\char`\\}
\def\PYZus{\char`\_}
\def\PYZob{\char`\{}
\def\PYZcb{\char`\}}
\def\PYZca{\char`\^}
\def\PYZam{\char`\&}
\def\PYZlt{\char`\<}
\def\PYZgt{\char`\>}
\def\PYZsh{\char`\#}
\def\PYZpc{\char`\%}
\def\PYZdl{\char`\$}
\def\PYZhy{\char`\-}
\def\PYZsq{\char`\'}
\def\PYZdq{\char`\"}
\def\PYZti{\char`\~}
% for compatibility with earlier versions
\def\PYZat{@}
\def\PYZlb{[}
\def\PYZrb{]}
\makeatother


    % Exact colors from NB
    \definecolor{incolor}{rgb}{0.0, 0.0, 0.5}
    \definecolor{outcolor}{rgb}{0.545, 0.0, 0.0}



    
    % Prevent overflowing lines due to hard-to-break entities
    \sloppy 
    % Setup hyperref package
    \hypersetup{
      breaklinks=true,  % so long urls are correctly broken across lines
      colorlinks=true,
      urlcolor=urlcolor,
      linkcolor=linkcolor,
      citecolor=citecolor,
      }
    % Slightly bigger margins than the latex defaults
    
    \geometry{verbose,tmargin=1in,bmargin=1in,lmargin=1in,rmargin=1in}
    
    

    \begin{document}
    
    
    \maketitle
    
    

    
    \hypertarget{githubux7b2cux4e00ux8bb2}{%
\section{Github第一讲}\label{githubux7b2cux4e00ux8bb2}}
注:此文档参考了
https://www.liaoxuefeng.com
http://www.runoob.com/w3cnote/git-guide.html
著作权归原作者所有
本人只做文档学习之用,侵删
    如果你是一枚Coder,但是你不知道Github,那么我觉的你就不是一个菜鸟级别的Coder,因为你压根不是真正Coder,你只是一个Code搬运工。------github简明教程

    \hypertarget{ux4ec0ux4e48ux662f-github}{%
\subsection{什么是 Github?}\label{ux4ec0ux4e48ux662f-github}}

    github是一个基于git的代码托管平台,付费用户可以建私人仓库,我们一般的免费用户只能使用公共仓库,也就是代码要公开。

    \hypertarget{ux5f00ux59cbux51c6ux5907ux6761ux4ef6}{%
\subsection{开始准备条件}\label{ux5f00ux59cbux51c6ux5907ux6761ux4ef6}}

    \begin{enumerate}
\def\labelenumi{\arabic{enumi}.}
\tightlist
\item
  github官网注册账号
\item
  Create a New
  Repository,填好名称后Create,之后会出现一些仓库的配置信息
\item
  Git安装
\end{enumerate}

    \hypertarget{ux914dux7f6egit}{%
\subsection{配置Git}\label{ux914dux7f6egit}}

    安装完成后,还需要最后一步设置,在命令行输入:
$ git config --global user.name "Your Name"
$ git config --global user.email "email@example.com"
    因为Git是分布式版本控制系统,所以,每个机器都必须自报家门:你的名字和Email地址。

    注意\texttt{git\ config}命令的\texttt{-\/-global}参数,用了这个参数,表示你这台机器上所有的Git仓库都会使用这个配置,当然也可以对某个仓库指定不同的用户名和Email地址。

    \hypertarget{ux521bux5efaux7248ux672cux5e93}{%
\subsection{创建版本库}\label{ux521bux5efaux7248ux672cux5e93}}

    什么是版本库呢?版本库又名仓库,英文名repository,你可以简单理解成一个目录,这个目录里面的所有文件都可以被Git管理起来,每个文件的修改、删除,Git都能跟踪,以便任何时刻都可以追踪历史,或者在将来某个时刻可以``还原''。而且,创建一个版本库非常简单。

    首先,选择一个合适的地方,创建一个空目录:
$ mkdir github_learning
$ cd github_learning
    \texttt{!!!}如果你使用Windows系统,为了避免遇到各种莫名其妙的问题,请确保目录名(包括父目录)不包含中文。

    第二步,通过\texttt{git\ init}命令把这个目录变成Git可以管理的仓库:

    瞬间Git就把仓库建好了,而且告诉你是一个空的仓库(empty Git
repository),细心的读者可以发现当前目录下多了一个\texttt{.git}的目录,这个目录是Git来跟踪管理版本库的,没事千万不要手动修改这个目录里面的文件,不然改乱了,就把Git仓库给破坏了。

    如果你没有看到\texttt{.git}目录,那是因为这个目录默认是隐藏的,用\texttt{ls\ -ah}命令就可以看见。

    第三步,把文件添加到版本库

    建议你下载\texttt{Notepad++}代替记事本,不但功能强大,而且免费!记得把Notepad++的默认编码设置为\texttt{UTF-8\ without\ BOM}即可:

    言归正传,现在我们编写一个\texttt{readme.txt}文件,内容随意

    一定要放到\texttt{github\_learning}目录下(子目录也行),因为这是一个Git仓库,放到其他地方Git再厉害也找不到这个文件。

把一个文件放到Git仓库只需要两步。

    第一步,用命令\texttt{git\ add}告诉Git,把文件添加到仓库:
$ git add readme.txt
    执行上面的命令,没有任何显示,这就对了,Unix的哲学是``没有消息就是好消息'',说明添加成功。

    第二步,用命令\texttt{git\ commit}告诉Git,把文件提交到仓库:
$ git commit -m "wrote a readme file"
    简单解释一下\texttt{git\ commit}命令,\texttt{-m}后面输入的是本次提交的说明,可以输入任意内容,当然最好是有意义的,这样你就能从历史记录里方便地找到改动记录。

    \texttt{git\ commit}命令执行成功后会告诉你,\texttt{1\ file\ changed}:1个文件被改动(我们新添加的readme.txt文件);\texttt{2\ insertions}:插入了两行内容(readme.txt有两行内容)。

    为什么Git添加文件需要add,commit一共两步呢?因为commit可以一次提交很多文件,所以你可以多次add不同的文件,比如:
$ git add file1.txt
$ git add file2.txt file3.txt
$ git commit -m "add 3 files."
    \hypertarget{ux65f6ux5149ux673aux7a7fux68ad}{%
\subsection{时光机穿梭}\label{ux65f6ux5149ux673aux7a7fux68ad}}

    我们已经成功地添加并提交了一个readme.txt文件,现在,是时候继续工作了,于是,我们继续修改readme.txt文件,改成如下内容:
Git is a distributed version control system.
Git is free software.
    现在,运行\texttt{git\ status}命令看看结果:
$ git status
On branch master
Changes not staged for commit:
  (use "git add <file>..." to update what will be committed)
  (use "git checkout -- <file>..." to discard changes in working directory)

    modified:   readme.txt

no changes added to commit (use "git add" and/or "git commit -a")
    \texttt{git\ status}命令可以让我们时刻掌握仓库当前的状态,上面的命令输出告诉我们,readme.txt被修改过了,但还没有准备提交的修改。

    虽然Git告诉我们\texttt{readme.txt}被修改了,但如果能看看具体修改了什么内容,自然是很好的。比如你休假两周从国外回来,第一天上班时,已经记不清上次怎么修改的\texttt{readme.txt},所以,需要用\texttt{git\ diff}这个命令看看:
$ git diff readme.txt 
diff --git a/readme.txt b/readme.txt
index 46d49bf..9247db6 100644
--- a/readme.txt
+++ b/readme.txt
@@ -1,2 +1,2 @@
-Git is a version control system.
+Git is a distributed version control system.
 Git is free software.
    \texttt{git\ diff}顾名思义就是查看difference,显示的格式正是Unix通用的diff格式,可以从上面的命令输出看到,我们在第一行添加了一个\texttt{distributed}单词。

    知道了对\texttt{readme.txt}作了什么修改后,再把它提交到仓库就放心多了,提交修改和提交新文件是一样的两步,第一步是\texttt{git\ add}:
$ git add readme.txt
    同样没有任何输出。在执行第二步\texttt{git\ commit}之前,我们再运行\texttt{git\ status}看看当前仓库的状态:
$ git status
On branch master
Changes to be committed:
  (use "git reset HEAD <file>..." to unstage)

    modified:   readme.txt
    \texttt{git\ status}告诉我们,将要被提交的修改包括\texttt{readme.txt},下一步,就可以放心地提交了:
$ git commit -m "add distributed"
[master e475afc] add distributed
 1 file changed, 1 insertion(+), 1 deletion(-)
    提交后,我们再用\texttt{git\ status}命令看看仓库的当前状态:
$ git status
On branch master
nothing to commit, working tree clean
    Git告诉我们当前没有需要提交的修改,而且,工作目录是干净(working tree
clean)的。

    \hypertarget{ux7248ux672cux56deux9000}{%
\subsection{版本回退}\label{ux7248ux672cux56deux9000}}

    现在,你已经学会了修改文件,然后把修改提交到Git版本库,现在,再练习一次,修改readme.txt文件如下:
Git is a distributed version control system.
Git is free software distributed under the GPL.
    然后尝试提交:
$ git add readme.txt
$ git commit -m "append GPL"
[master 1094adb] append GPL
 1 file changed, 1 insertion(+), 1 deletion(-)
    Git也是一样,每当你觉得文件修改到一定程度的时候,就可以``保存一个快照'',这个快照在Git中被称为commit。一旦你把文件改乱了,或者误删了文件,还可以从最近的一个commit恢复,然后继续工作,而不是把几个月的工作成果全部丢失。

    现在,我们回顾一下\texttt{readme.txt}文件一共有几个版本被提交到Git仓库里了:

    版本1:wrote a readme file
Git is a version control system.
Git is free software.
    版本2:add distributed
Git is a distributed version control system.
Git is free software.
    版本3:append GPL
Git is a distributed version control system.
Git is free software distributed under the GPL.
    当然了,在实际工作中,我们脑子里怎么可能记得一个几千行的文件每次都改了什么内容,不然要版本控制系统干什么。版本控制系统肯定有某个命令可以告诉我们历史记录,在Git中,我们用\texttt{git\ log}命令查看:
$ git log
commit 1094adb7b9b3807259d8cb349e7df1d4d6477073 (HEAD -> master)
Author: Michael Liao <askxuefeng@gmail.com>
Date:   Fri May 18 21:06:15 2018 +0800

    append GPL

commit e475afc93c209a690c39c13a46716e8fa000c366
Author: Michael Liao <askxuefeng@gmail.com>
Date:   Fri May 18 21:03:36 2018 +0800

    add distributed

commit eaadf4e385e865d25c48e7ca9c8395c3f7dfaef0
Author: Michael Liao <askxuefeng@gmail.com>
Date:   Fri May 18 20:59:18 2018 +0800

    wrote a readme file
    \texttt{git\ log}命令显示从最近到最远的提交日志,我们可以看到3次提交,最近的一次是\texttt{append\ GPL},上一次是\texttt{add\ distributed},最早的一次是\texttt{wrote\ a\ readme\ file}。

    好了,现在我们启动时光穿梭机,准备把\texttt{readme.txt}回退到上一个版本,也就是\texttt{add\ distributed}的那个版本,怎么做呢?

    首先,Git必须知道当前版本是哪个版本,在Git中,用\texttt{HEAD}表示当前版本,也就是最新的提交\texttt{1094adb...}(注意我的提交ID和你的肯定不一样),上一个版本就是\texttt{HEAD\^{}},上上一个版本就是\texttt{HEAD\^{}\^{}},当然往上100个版本写100个\^{}比较容易数不过来,所以写成\texttt{HEAD\textasciitilde{}100}。

    现在,我们要把当前版本\texttt{append\ GPL}回退到上一个版本\texttt{add\ distributed},就可以使用\texttt{git\ reset}命令:
$ git reset --hard HEAD^
HEAD is now at e475afc add distributed
    --hard参数有啥意义?这个后面再讲,现在你先放心使用。

    看看\texttt{readme.txt}的内容是不是版本\texttt{add\ distributed}:
$ cat readme.txt
Git is a distributed version control system.
Git is free software.
    果然被还原了。

    还可以继续回退到上一个版本\texttt{wrote\ a\ readme\ file},不过且慢,然我们用\texttt{git\ log}再看看现在版本库的状态:
$ git log
commit e475afc93c209a690c39c13a46716e8fa000c366 (HEAD -> master)
Author: Michael Liao <askxuefeng@gmail.com>
Date:   Fri May 18 21:03:36 2018 +0800

    add distributed

commit eaadf4e385e865d25c48e7ca9c8395c3f7dfaef0
Author: Michael Liao <askxuefeng@gmail.com>
Date:   Fri May 18 20:59:18 2018 +0800

    wrote a readme file
    最新的那个版本\texttt{append\ GPL}已经看不到了!好比你从21世纪坐时光穿梭机来到了19世纪,想再回去已经回不去了,肿么办?

    办法其实还是有的,只要上面的命令行窗口还没有被关掉,你就可以顺着往上找啊找啊,找到那个\texttt{append\ GPL}的\texttt{commit\ id}是\texttt{1094adb...},于是就可以指定回到未来的某个版本:
$ git reset --hard 1094a
HEAD is now at 83b0afe append GPL
    版本号没必要写全,前几位就可以了,Git会自动去找。当然也不能只写前一两位,因为Git可能会找到多个版本号,就无法确定是哪一个了。

    再小心翼翼地看看\texttt{readme.txt}的内容:
$ cat readme.txt
Git is a distributed version control system.
Git is free software distributed under the GPL.
    果然,我胡汉三又回来了。

    Git的版本回退速度非常快,因为Git在内部有个指向当前版本的\texttt{HEAD}指针,当你回退版本的时候,Git仅仅是把\texttt{HEAD}从指向\texttt{append\ GPL}:

    \begin{figure}
\centering
\includegraphics{attachment:image.png}
\caption{image.png}
\end{figure}

    改为指向\texttt{add\ distributed}:

    \begin{figure}
\centering
\includegraphics{attachment:image.png}
\caption{image.png}
\end{figure}

    然后顺便把工作区的文件更新了。所以你让\texttt{HEAD}指向哪个版本号,你就把当前版本定位在哪。

    现在,你回退到了某个版本,关掉了电脑,第二天早上就后悔了,想恢复到新版本怎么办?找不到新版本的\texttt{commit\ id}怎么办?

    在Git中,总是有后悔药可以吃的。当你用\texttt{\$\ git\ reset\ -\/-hard\ HEAD\^{}}回退到\texttt{add\ distributed}版本时,再想恢复到\texttt{append\ GPL},就必须找到\texttt{append\ GPL}的\texttt{commit\ id}。Git提供了一个命令\texttt{git\ reflog}用来记录你的每一次命令:
$ git reflog
e475afc HEAD@{1}: reset: moving to HEAD^
1094adb (HEAD -> master) HEAD@{2}: commit: append GPL
e475afc HEAD@{3}: commit: add distributed
eaadf4e HEAD@{4}: commit (initial): wrote a readme file
    终于舒了口气,从输出可知,\texttt{append\ GPL}的\texttt{commit\ id}是\texttt{1094adb},现在,你又可以乘坐时光机回到未来了。

    \hypertarget{ux5de5ux4f5cux533aux548cux6682ux5b58ux533a}{%
\subsection{工作区和暂存区}\label{ux5de5ux4f5cux533aux548cux6682ux5b58ux533a}}

    工作区(Working Directory)

    就是你在电脑里能看到的目录,比如我的\texttt{learngit}文件夹就是一个工作区:

    版本库(Repository)

    工作区有一个隐藏目录\texttt{.git},这个不算工作区,而是Git的版本库。

    Git的版本库里存了很多东西,其中最重要的就是称为\texttt{stage}(或者叫\texttt{index})的暂存区,还有Git为我们自动创建的第一个分支\texttt{master},以及指向\texttt{master}的一个指针叫\texttt{HEAD}。

    \begin{figure}
\centering
\includegraphics{attachment:image.png}
\caption{image.png}
\end{figure}

    前面讲了我们把文件往Git版本库里添加的时候,是分两步执行的:

    第一步是用\texttt{git\ add}把文件添加进去,实际上就是把文件修改添加到暂存区;

    二步是用\texttt{git\ commit}提交更改,实际上就是把暂存区的所有内容提交到当前分支。

    因为我们创建Git版本库时,Git自动为我们创建了唯一一个\texttt{master}分支,所以,现在,\texttt{git\ commit}就是往\texttt{master}分支上提交更改。

    你可以简单理解为,需要提交的文件修改通通放到暂存区,然后,一次性提交暂存区的所有修改。

    暂存区是Git非常重要的概念,弄明白了暂存区,就弄明白了Git的很多操作到底干了什么。

    \hypertarget{ux7ba1ux7406ux4feeux6539}{%
\subsection{管理修改}\label{ux7ba1ux7406ux4feeux6539}}

    Git跟踪并管理的是修改,而非文件。

    你会问,什么是修改?比如你新增了一行,这就是一个修改,删除了一行,也是一个修改,更改了某些字符,也是一个修改,删了一些又加了一些,也是一个修改,甚至创建一个新文件,也算一个修改。

    每次修改,如果不用\texttt{git\ add}到暂存区,那就不会加入到\texttt{commit}中。

    \hypertarget{ux64a4ux9500ux4feeux6539}{%
\subsection{撤销修改}\label{ux64a4ux9500ux4feeux6539}}

    \texttt{git\ checkout\ -\/-\ file}可以丢弃工作区的修改:
$ git checkout -- readme.txt
    命令\texttt{git\ checkout\ -\/-\ readme.txt}意思就是,把\texttt{readme.txt}文件在工作区的修改全部撤销,这里有两种情况:

一种是\texttt{readme.txt}自修改后还没有被放到暂存区,现在,撤销修改就回到和版本库一模一样的状态;

一种是\texttt{readme.txt}已经添加到暂存区后,又作了修改,现在,撤销修改就回到添加到暂存区后的状态。

总之,就是让这个文件回到最近一次\texttt{git\ commit}或\texttt{git\ add}时的状态。

    \texttt{git\ checkout\ -\/-\ file}命令中的\texttt{-\/-}很重要,没有\texttt{-\/-},就变成了``切换到另一个分支''的命令,我们在后面的分支管理中会再次遇到\texttt{git\ checkout}命令。

    \texttt{git\ reset}命令既可以回退版本,也可以把暂存区的修改回退到工作区。当我们用\texttt{HEAD}时,表示最新的版本。

    场景1:当你改乱了工作区某个文件的内容,想直接丢弃工作区的修改时,用命令\texttt{git\ checkout\ -\/-\ file。}

场景2:当你不但改乱了工作区某个文件的内容,还添加到了暂存区时,想丢弃修改,分两步,第一步用命令\texttt{git\ reset\ HEAD\ \textless{}file\textgreater{}},就回到了场景1,第二步按场景1操作。

场景3:已经提交了不合适的修改到版本库时,想要撤销本次提交,参考版本回退一节,不过前提是没有推送到远程库。

    \hypertarget{ux5220ux9664ux6587ux4ef6}{%
\subsection{删除文件}\label{ux5220ux9664ux6587ux4ef6}}

    一般情况下,你通常直接在文件管理器中把没用的文件删了,或者用\texttt{rm}命令删了:

    这个时候,Git知道你删除了文件,因此,工作区和版本库就不一致了,\texttt{git\ status}命令会立刻告诉你哪些文件被删除了:
$ git status
On branch master
Changes not staged for commit:
  (use "git add/rm <file>..." to update what will be committed)
  (use "git checkout -- <file>..." to discard changes in working directory)

    deleted:    test.txt

no changes added to commit (use "git add" and/or "git commit -a")
    现在你有两个选择,一是确实要从版本库中删除该文件,那就用命令\texttt{git\ rm}删掉,并且\texttt{git\ commit}:
$ git rm test.txt
rm 'test.txt'

$ git commit -m "remove test.txt"
[master d46f35e] remove test.txt
 1 file changed, 1 deletion(-)
 delete mode 100644 test.txt
    现在,文件就从版本库中被删除了。

    另一种情况是删错了,因为版本库里还有呢,所以可以很轻松地把误删的文件恢复到最新版本:
$ git checkout -- test.txt
    \texttt{git\ checkout}其实是用版本库里的版本替换工作区的版本,无论工作区是修改还是删除,都可以``一键还原''。

    \hypertarget{ux8fdcux7a0bux4ed3ux5e93}{%
\subsection{远程仓库}\label{ux8fdcux7a0bux4ed3ux5e93}}

    实际情况往往是这样,找一台电脑充当服务器的角色,每天24小时开机,其他每个人都从这个``服务器''仓库克隆一份到自己的电脑上,并且各自把各自的提交推送到服务器仓库里,也从服务器仓库中拉取别人的提交。

    只要注册一个GitHub账号,就可以免费获得Git远程仓库。

    由于你的本地Git仓库和GitHub仓库之间的传输是通过SSH加密的,所以,需要一点设置:

    第1步:创建SSH
Key。在用户主目录下,看看有没有\texttt{.ssh}目录,如果有,再看看这个目录下有没有\texttt{id\_rsa}和\texttt{id\_rsa.pub}这两个文件,如果已经有了,可直接跳到下一步。如果没有,打开Shell(Windows下打开Git
Bash),创建SSH Key:
$ ssh-keygen -t rsa -C "youremail@example.com"
    你需要把邮件地址换成你自己的github注册邮件地址,然后一路回车,使用默认值即可,由于这个Key也不是用于军事目的,所以也无需设置密码。

    如果一切顺利的话,可以在用户主目录里找到\texttt{.ssh}目录,里面有\texttt{id\_rsa}和\texttt{id\_rsa.pub}两个文件,这两个就是SSH
Key的秘钥对,\texttt{id\_rsa}是私钥,不能泄露出去,\texttt{id\_rsa.pub}是公钥,可以放心地告诉任何人。

    第2步:登陆GitHub,打开``Account settings'',``SSH Keys''页面:

    然后,点``Add SSH
Key'',填上任意Title,在Key文本框里粘贴\texttt{id\_rsa.pub}文件的内容:

    点``Add Key'',你就应该看到已经添加的Key:

    为什么GitHub需要SSH
Key呢?因为GitHub需要识别出你推送的提交确实是你推送的,而不是别人冒充的,而Git支持SSH协议,所以,GitHub只要知道了你的公钥,就可以确认只有你自己才能推送。

    当然,GitHub允许你添加多个Key。假定你有若干电脑,你一会儿在公司提交,一会儿在家里提交,只要把每台电脑的Key都添加到GitHub,就可以在每台电脑上往GitHub推送了。

    最后友情提示,在GitHub上免费托管的Git仓库,任何人都可以看到喔(但只有你自己才能改)。所以,不要把敏感信息放进去。

    \hypertarget{ux6dfbux52a0ux8fdcux7a0bux5e93}{%
\subsection{添加远程库}\label{ux6dfbux52a0ux8fdcux7a0bux5e93}}

    现在的情景是,你已经在本地创建了一个Git仓库后,又想在GitHub创建一个Git仓库,并且让这两个仓库进行远程同步,这样,GitHub上的仓库既可以作为备份,又可以让其他人通过该仓库来协作,真是一举多得。

    首先,登陆GitHub,然后,在右上角找到``Create a new
repo''按钮,创建一个新的仓库:

    在Repository
name填入\texttt{github\_learning},其他保持默认设置,不要勾选\texttt{README.md}选项,点击``Create
repository''按钮,就成功地创建了一个新的Git仓库:

    目前,在GitHub上的这个\texttt{github\_learning}仓库还是空的,GitHub告诉我们,可以从这个仓库克隆出新的仓库,也可以把一个已有的本地仓库与之关联,然后,把本地仓库的内容推送到GitHub仓库。

    现在,我们根据GitHub的提示,在本地的\texttt{github\_learning}仓库下运行命令:
$ git remote add origin git@github.com:your_github_account/your_repository.git
    请千万注意,把上面的\texttt{your\_github\_account}替换成你自己的GitHub账户名,\texttt{your\_repository}替换成自己的仓库名,否则,你在本地关联的就是我的远程库,关联没有问题,但是你以后推送是推不上去的,因为你的SSH
Key公钥不在我的账户列表中。

    添加后,远程库的名字就是\texttt{origin},这是Git默认的叫法,也可以改成别的,但是\texttt{origin}这个名字一看就知道是远程库。

    下一步,就可以把本地库的所有内容推送到远程库上:
$ git push -u origin master
    把本地库的内容推送到远程,用\texttt{git\ push}命令,实际上是把当前分支\texttt{master}推送到远程。

    由于远程库是空的,我们第一次推送\texttt{master}分支时,加上了\texttt{-u}参数,Git不但会把本地的master分支内容推送的远程新的master分支,还会把本地的master分支和远程的master分支关联起来,在以后的推送或者拉取时就可以简化命令。

    推送成功后,可以立刻在GitHub页面中看到远程库的内容已经和本地一模一样:

    从现在起,只要本地作了提交,就可以通过命令:
$ git push origin master
    把本地master分支的最新修改推送至GitHub,现在,你就拥有了真正的分布式版本库!

    \hypertarget{sshux8b66ux544a}{%
\subsection{SSH警告}\label{sshux8b66ux544a}}

    当你第一次使用Git的\texttt{clone}或者\texttt{push}命令连接GitHub时,会得到一个警告:
The authenticity of host 'github.com (xx.xx.xx.xx)' can't be established.
RSA key fingerprint is xx.xx.xx.xx.xx.
Are you sure you want to continue connecting (yes/no)?
    这是因为Git使用SSH连接,而SSH连接在第一次验证GitHub服务器的Key时,需要你确认GitHub的Key的指纹信息是否真的来自GitHub的服务器,输入\texttt{yes}回车即可。

    Git会输出一个警告,告诉你已经把GitHub的Key添加到本机的一个信任列表里了:
Warning: Permanently added 'github.com' (RSA) to the list of known hosts.
    这个警告只会出现一次,后面的操作就不会有任何警告了。

    如果你实在担心有人冒充GitHub服务器,输入\texttt{yes}前可以对照GitHub的RSA
Key的指纹信息是否与SSH连接给出的一致。

    要关联一个远程库,使用命令\texttt{git\ remote\ add\ origin\ git@server-name:path/repo-name.git};

关联后,使用命令\texttt{git\ push\ -u\ origin\ master}第一次推送master分支的所有内容;

此后,每次本地提交后,只要有必要,就可以使用命令\texttt{git\ push\ origin\ master}推送最新修改;

分布式版本系统的最大好处之一是在本地工作完全不需要考虑远程库的存在,也就是有没有联网都可以正常工作,而SVN在没有联网的时候是拒绝干活的!当有网络的时候,再把本地提交推送一下就完成了同步,真是太方便了!

    \hypertarget{ux4eceux8fdcux7a0bux5e93ux514bux9686}{%
\subsection{从远程库克隆}\label{ux4eceux8fdcux7a0bux5e93ux514bux9686}}

    上次我们讲了先有本地库,后有远程库的时候,如何关联远程库。

    现在,假设我们从零开发,那么最好的方式是先创建远程库,然后,从远程库克隆。

    首先,登陆GitHub,创建一个新的仓库,名字叫\texttt{gitskills}:

    我们勾选\texttt{Initialize\ this\ repository\ with\ a\ README},这样GitHub会自动为我们创建一个\texttt{README.md}文件。创建完毕后,可以看到\texttt{README.md}文件:

    现在,远程库已经准备好了,下一步是用命令\texttt{git\ clone}克隆一个本地库:
$ git clone git@github.com:michaelliao/gitskills.git
Cloning into 'gitskills'...
remote: Counting objects: 3, done.
remote: Total 3 (delta 0), reused 0 (delta 0), pack-reused 3
Receiving objects: 100% (3/3), done.
    注意把Git库的地址换成你自己的,然后进入\texttt{gitskills}目录看看,已经有\texttt{README.md}文件了:
$ cd gitskills
$ ls
README.md
    如果有多个人协作开发,那么每个人各自从远程克隆一份就可以了。

    你也许还注意到,GitHub给出的地址不止一个,还可以用\texttt{https://github.com/michaelliao/gitskills.git}这样的地址。实际上,Git支持多种协议,默认的\texttt{git://}使用\texttt{ssh},但也可以使用\texttt{https}等其他协议。

    使用\texttt{https}除了速度慢以外,还有个最大的麻烦是每次推送都必须输入口令,但是在某些只开放http端口的公司内部就无法使用\texttt{ssh}协议而只能用\texttt{https}。


    % Add a bibliography block to the postdoc
    
    
    
    \end{document}
